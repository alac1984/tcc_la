\section{INTRODUÇÃO}

A história do ensino de línguas, como aponta \citeonline[p. 3]{richards2001}, é caracterizada pela busca de formas mais eficientes de ensinar uma segunda língua. Embora possa-se dizer que esse campo de conhecimento possui séculos de existência — inicialmente dominados pela tradição que herdou do ensino de latim os métodos e procedimentos de ensino de línguas — foi, segundo \citeonline[p. 15]{richards2001}no período entre os anos 1950 e 1980 que a busca por formas mais eficientes de ensinar uma segunda língua intensificou-se. Isso se deu, em grande parte, pela importância que a língua inglesa adquiriu como meio de instrução na sociedade nesse período, marcada por um intenso processo de globalização e cujas comunicações foram amplamente transformadas pelo advento da Tecnologia da Informação e Comunicação (TIC), aumentando portanto a demanda por eficiência nesse ramo. As transformações nos campos da linguística, linguística aplicada, psicologia do desenvolvimento e psicologia da educação, como apresentadas em \citeonline{weiner2003}, também foram responsáveis pelo embasamento teórico de uma série de novas formas de pensar o ensino de línguas, além de retirar tração e suportar antigas formas de pensar esse mesmo campo. A união desses aspectos supracitados fomentou então um contexto ideal para o surgimento de novas formas de realizar a prática docente na área de línguas. 

Nesse contexto do surgimento de novos métodos de ensino de línguas surgiu a Abordagem Lexical, desenvolvida por Michael Lewis no início dos anos 1990. O texto seminal de Lewis surgiu como uma revolução no campo da linguística aplicada, sobretudo no ensino de língua estrangeira. 

Este trabalho se propõe a realizar uma pesquisa bibliográfica, que, para \citeonline{mazucato2018} é  um método que consiste no levantamento, seleção e análise de publicações disponíveis publicamente, como livros, artigos científicos e documentos eletrônicos, para construir um panorama compreensivo sobre um determinado tema ou questão de pesquisa. O objetivo principal deste trabalho é explorar a Abordagem Lexical, enfocando suas implicações teóricas e práticas para o ensino de línguas. A análise se concentrará em como essa abordagem influencia os métodos de ensino, a aquisição de segunda língua e a preparação dos professores, além de investigar as críticas e as potencialidades identificadas na literatura especializada. Ao elucidar estes aspectos, o trabalho visa contribuir para uma melhor compreensão de como a Abordagem Lexical pode ser aplicada de maneira eficaz no contexto educacional contemporâneo.


\section{CONCEITOS IMPORTANTES}

Antes de mergulharmos na exploração detalhada da Abordagem Lexical, é imperativo estabelecer uma fundação sólida sobre a qual possamos construir nossa discussão e análise. A compreensão das diversas teorias e práticas que informam o ensino de línguas não apenas contextualiza nossa investigação, mas também enriquece nosso entendimento das implicações práticas no campo do ensino de línguas. Portanto, nesta seção, delinearemos uma série de conceitos fundamentais que são pedras angulares no estudo da didática de línguas.

\subsection{ABORDAGEM, MÉTODO E TÉCNICA}

Em nossa argumentação vamos utilizar o modelo de classificação criada em 1963 pelo professor Edward Mason Anthony Jr. da Universidade de Michigan, que, segundo \citeonline{richards2001}, visava classificar uma desnorteadora variedade de termos usados para descrever as atividades nas quais se engajam e as crenças que possuem os professores de línguas. Em seu modelo Anthony propôs três importantes conceitualizações: a de abordagem, a de método e a de técnica.

Segundo Anthony uma abordagem é “um conjunto de suposições correlativas que lidam com a natureza do ensino e aprendizagem de línguas.” \cite[p. 19]{richards2001} A abordagem, ainda para Anthony, "é axiomática”. As abordagens estão, portanto, em nível mais estratégico. Segundo Richard e Rodgers (2001) as abordagens buscam seus fundamentos teóricos nas diferentes visões acerca da natureza da linguagem que os estudos linguísticos nos propõem. Em seu trabalho seminal, Lewis expande o conceito de abordagem e a apresenta como "é um conjunto integrado de crenças teóricas e práticas, que incorpora tanto o currículo quanto o método" \cite[p. 14]{lewis1993}.

Já o método, segundo Anthony \cite{richards2001}, “é um plano geral para a apresentação ordenada de material de linguagem, nenhuma parte do qual contradiz, e tudo baseado na abordagem selecionada.” O autor completa dizendo que “uma abordagem é axiomática, um método é procedimental” \cite[p. 19]{richards2001}. Podemos compreender que o papel do método nessa estrutura de elementos do ensino de línguas é menos estratégico e abstrato que o da abordagem, mas não tão tático e concreto quanto a técnica. Um método estrutura os princípios norteadores da abordagem e os converte em diretrizes práticas. Também nesse ponto \citeonline[p. 15]{lewis1993} apresenta o método como estratégias de sala de aula, dando a este um caráter mais prático que a abordagem.

E por último temos a técnica, que, como apresentado por \citeonline{richards2001}, “é implementacional - aquilo que realmente ocorre em uma sala de aula. É um truque, estratagema ou artifício específico usado para atingir um objetivo imediato. As técnicas devem ser consistentes com um método e, portanto, também em harmonia com uma abordagem.” Compreendemos, portanto, que a técnica é o processo tático per si, a atividade que é ministrada aos alunos e que carrega em si a estrutura do método, e este carrega em si os princípios da abordagem. Dessa forma, dizemos que os elementos dessa classificação apresentam-se de forma hierárquica, onde a abordagem fornece os pressupostos para o método, e este fornece a estrutura para a técnica.

Para finalizar esta seção, é crucial enfatizar que nosso foco subsequente será explorar a Abordagem Lexical primordialmente sob a perspectiva de uma abordagem, conforme definido por Edward Mason Anthony Jr. Dessa maneira, nosso objetivo não será detalhar técnicas individuais de aplicação ou métodos específicos, mas sim investigar e discutir os princípios teóricos e práticas integradas que fundamentam esta abordagem no ensino de línguas. Ao abordar a Abordagem Lexical como uma estrutura conceitual abrangente, pretendemos destacar como suas premissas orientam as práticas pedagógicas e influenciam a maneira como o ensino e a aprendizagem de línguas são percebidos e implementados. Esse entendimento mais estratégico nos permitirá avaliar a relevância e a eficácia da Abordagem Lexical no contexto contemporâneo do ensino de idiomas, contribuindo assim para uma compreensão mais profunda e aplicada deste influente modelo pedagógico.

\subsection{SOBRE A NATUREZA DA LÍNGUA}

Segundo \citeonline{richards2001}, pelo menos três visões teóricas da linguagem e sobre a natureza da proficiência em línguas implícita ou explicitamente embasam as diferentes abordagens e métodos de ensino de línguas. São elas: a visão estrutural, a visão funcional e a visão interacional.

Os autores as definem da seguinte forma:

\begin{citar}
A primeira, e a mais tradicional das três, é a visão estrutural, a visão de que a linguagem é um sistema de elementos estruturalmente relacionados para a codificação do significado. O alvo da aprendizagem de línguas é visto como o domínio dos elementos deste sistema (...). A segunda visão da linguagem é a visão funcional, a visão de que a linguagem é um veículo para a expressão do significado funcional. (...) Essa teoria enfatiza a dimensão semântica e comunicativa, em vez de meramente as características gramaticais da linguagem, e leva a uma especificação e organização do conteúdo do ensino de línguas por categorias de significado e função, e não por elementos de estrutura e gramática. (...) A terceira visão da linguagem pode ser chamada de visão interacional. Ela vê a linguagem como um veículo para a realização das relações interpessoais e para a realização das transações sociais entre os indivíduos. A linguagem é vista como uma ferramenta para a criação e manutenção das relações sociais. As áreas de investigação que estão sendo elaboradas no desenvolvimento de abordagens interacionais para o ensino de línguas incluem análise de interação, análise de conversação e etnometodologia. \cite{richards2001}
\end{citar}

No contexto deste trabalho, compreender as visões estrutural, funcional e interacional da linguagem é fundamental para situar a Abordagem Lexical dentro de um arcabouço teórico mais amplo sobre a natureza da língua e do ensino de idiomas. Essas três perspectivas nos ajudam a entender como diferentes abordagens e métodos de ensino refletem concepções variadas sobre o que é a linguagem e como ela deve ser aprendida e ensinada. Em particular, a Abordagem Lexical, ao enfatizar o aprendizado de "chunks" lexicais e a importância do léxico na comunicação, dialoga especialmente com a visão funcional, na qual a linguagem serve primariamente como um veículo para a expressão de significados funcionais e pragmáticos. No entanto, ela também incorpora elementos das visões estrutural e interacional, ao reconhecer a estruturação do léxico dentro da linguagem e sua utilização nas interações sociais. Assim, ao explorar a Abordagem Lexical em nosso TCC, buscaremos destacar como ela se alinha e se diferencia dessas visões teóricas, proporcionando uma compreensão mais completa de sua aplicabilidade e eficácia no ensino de línguas.

\section{PRINCÍPIOS CHAVE DA ABORDAGEM LEXICAL}

Ao desbravar o campo do ensino de línguas com sua obra seminal, Michael Lewis não apenas propôs uma nova perspectiva, mas também nos apresentou um conjunto de princípios fundamentais que servem como a espinha dorsal da Abordagem Lexical. Estes princípios não são apenas diretrizes; eles são revelações que desafiam e expandem nossa compreensão de como as línguas são aprendidas e ensinadas. Em sintonia com o método pioneiro de Lewis, esta seção de nosso trabalho buscará articular esses princípios chave, que, juntos, tecem o vasto e complexo tapeçar do léxico como o coração pulsante da aprendizagem da língua. \citeonline[p. 4]{torres_ramirez2012} define essa lista de princípios chave de "sumário prático de descobertas" das investigações de Lewis sobre a aquisição de primeira língua.

Ao adotarmos essa mesma estratégia elucidativa, nossa discussão não se limitará a uma mera exposição; aspiramos elevar a importância desses princípios à medida que exploramos suas implicações profundas para a prática pedagógica contemporânea. Portanto, tal como Lewis delineou os contornos de sua abordagem com princípios que são ao mesmo tempo profundos e práticos, nosso artigo se propõe a desdobrar essas noções-chave, estabelecendo um sólido entendimento da Abordagem Lexical e sua aplicação transformadora no ensino de línguas.

\section{}
\section{}

\citeonline{torres_ramirez2012}
