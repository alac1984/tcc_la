\documentclass[
    a4paper,          % Tamanho da folha A4
    12pt,             % Tamanho da fonte 12pt
    section=TITLE,    % Todas as secoes devem ter caixa alta
    subsection=Title, % Todas as subsecoes devem ter caixa alta somente na primeira letra
    oneside,          % Usada para impressao em apenas uma face do papel
    english,          % Hifenizacoes em ingles
    spanish,          % Hifenizacoes em espanhol
    brazil,           % Ultimo idioma eh o idioma padrao do documento
    fleqn             % Comente esta linha :wse quiser centralizar as equacoes. Comente também a linha 65 abaixo
]{abntex2}

\counterwithout{section}{chapter}
% Importações de pacotes
\usepackage[utf8]{inputenc}                         % Acentuação direta
\usepackage[T1]{fontenc}                            % Codificação da fonte em 8 bits
\usepackage{graphicx}                               % Inserir figuras
\usepackage{amsfonts, amssymb, amsmath}             % Fonte e símbolos matemáticos
\usepackage{booktabs}                               % Comandos para tabelas
\usepackage{verbatim}                               % Texto é interpretado como escrito no documento
\usepackage{multirow, array}                        % Múltiplas linhas e colunas em tabelas
\usepackage{indentfirst}                            % Endenta o primeiro parágrafo de cada seção.
\usepackage{listings}                               % Utilizar codigo fonte no documento
\usepackage{xcolor}
\usepackage{microtype}                              % Para melhorias de justificação?
\usepackage[portuguese,ruled,lined]{algorithm2e}    % Escrever algoritmos
\usepackage{algorithmic}                            % Criar Algoritmos  
%\usepackage{float}                                 % Utilizado para criação de floats
\usepackage{amsgen}
\usepackage{lipsum}                                 % Usar a simulação de texto Lorem Ipsum
\usepackage{titlesec}                              % Permite alterar os títulos do documento
\titleformat*{\section}{\normalfont\Large\bfseries}
\usepackage{tocloft}                                % Permite alterar a formatação do Sumário
\usepackage{etoolbox}                               % Usado para alterar a fonte da Section no Sumário
\usepackage[nogroupskip,nonumberlist]{glossaries}   % Permite fazer o glossario. A apcao "sort=use" faz com que as siglas aparecam na lista conformse sao usadas no texto.

\usepackage[format=plain,justification=justified,skip=0pt,singlelinecheck = false,labelsep=colon]{caption}            % Altera o comportamento da tag caption. Algumas opcoes do caption so podem ser alternada no arquivo "antex2.cls, linhas 334 a 348.

%\usepackage[bottom]{footmisc}                      % Mantém as notas de rodapé sempre na mesma posição
%\usepackage{times}                                 % Usa a fonte Times
%%%%%%%%%%%%%%%%%%% AVISO %%%%%%%%%%%%%%%%%%%%%%%%%%%%%%%%%%%%%%%%
%descomente as duas linhas abaixo para alterar o texto de Times New Roman para Arial:

%\usepackage{helvet}
%\renewcommand{\familydefault}{\sfdefault}  % Usa a fonte Arial              
%%%%%%%%%%%%%%%%%%%%%%%%%%%%%%%%%%%%%%%%%%%%%%%%%%%%%%%%%%%%%%%%%%

\usepackage{mathptmx}         % Usa a fonte Times New Roman			
%\usepackage{lmodern}         % Usa a fonte Latin Modern
%\usepackage{subfig}          % Posicionamento de figuras
%\usepackage{scalefnt}        % Permite redimensionar tamanho da fonte
%\usepackage{color, colortbl} % Comandos de cores
%\usepackage{lscape}          % Permite páginas em modo "paisagem"
%\usepackage{ae, aecompl}     % Fontes de alta qualidade
%\usepackage{picinpar}        % Dispor imagens em parágrafos
%\usepackage{latexsym}        % Símbolos matemáticos
%\usepackage{upgreek}         % Fonte letras gregas
\usepackage{appendix}         % Gerar o apendice no final do documento
\usepackage{paracol}          % Criar paragrafos sem identacao
\usepackage{pdfpages}         % Incluir pdf no documento
\usepackage{amsmath}          % Usar equacoes matematicas

\makeglossaries % Organiza e gera a lista de abreviaturas, simbolos e glossario
\makeindex      % Gera o Indice do documento         

\renewcommand{\labelitemi}{\textendash} %Altera os marcadores de itemize para 

\title{O Léxico Gramatizado: Uma Exploração Bibliográfica da Abordagem Lexical}
\author{André Carvalho} 
\date{2024\\ Abril}


\newenvironment{citar}{
  \begin{list}{}{
    \setlength{\leftmargin}{4cm}  % recuo à esquerda
    \setlength{\rightmargin}{0cm} % recuo à direita, se necessário
    \setlength{\parskip}{0cm}     % espaçamento entre parágrafos
  }
  \item[]
}{
  \end{list}
}

\begin{document}	

  \begin{flushright}
    \textbf{André Carvalho}

    \textbf{Nome do Professor}
  \end{flushright}

  \begin{center}
      \Large \textbf{RESUMO}
  \end{center}

  \noindent // TODO Aqui é apresentado o trabalho de forma sucinta (até 250 palavras, de 8 a 10 linhas, apenas um parágrafo e espaçamento simples), informando ao leitor finalidade, metodologia, resultados e conclusão do documento, de tal forma que este possa, inclusive, dispensar a consulta ao original se julgado que o conteúdo não é de seu interesse. O resumo deve ser composto de uma sequência de frases concisas, afirmativas, e não de enumeração de tópicos, lembrando que é necessário o uso de parágrafo único. A primeira frase deve ser significativa, explicando o tema principal do documento. A seguir, deve-se indicar a informação sobre a categoria do tratamento (memória, estudo de caso, análise da situação etc.).

  \noindent \textbf{PALAVRAS-CHAVE:} Termos e Palavras 1. Termos e Palavras 2. Termos e Palavras 3.

  \section{INTRODUÇÃO}

A história do ensino de línguas, como aponta \citeonline[p. 3]{richards2001}, é caracterizada pela busca de formas mais eficientes de ensinar uma segunda língua. Embora possa-se dizer que esse campo de conhecimento possui séculos de existência — inicialmente dominados pela tradição que herdou do ensino de latim os métodos e procedimentos de ensino de línguas — foi, segundo \citeonline[p. 15]{richards2001}no período entre os anos 1950 e 1980 que a busca por formas mais eficientes de ensinar uma segunda língua intensificou-se. Isso se deu, em grande parte, pela importância que a língua inglesa adquiriu como meio de instrução na sociedade nesse período, marcada por um intenso processo de globalização e cujas comunicações foram amplamente transformadas pelo advento da Tecnologia da Informação e Comunicação (TIC), aumentando portanto a demanda por eficiência nesse ramo. As transformações nos campos da linguística, linguística aplicada, psicologia do desenvolvimento e psicologia da educação, como apresentadas em \citeonline{weiner2003}, também foram responsáveis pelo embasamento teórico de uma série de novas formas de pensar o ensino de línguas, além de retirar tração e suportar antigas formas de pensar esse mesmo campo. A união desses aspectos supracitados fomentou então um contexto ideal para o surgimento de novas formas de realizar a prática docente na área de línguas. 

Nesse contexto do surgimento de novos métodos de ensino de línguas surgiu a Abordagem Lexical, desenvolvida por Michael Lewis no início dos anos 1990. O texto seminal de Lewis surgiu como uma revolução no campo da linguística aplicada, sobretudo no ensino de língua estrangeira. 

Este trabalho se propõe a realizar uma pesquisa bibliográfica, que, para \citeonline{mazucato2018} é  um método que consiste no levantamento, seleção e análise de publicações disponíveis publicamente, como livros, artigos científicos e documentos eletrônicos, para construir um panorama compreensivo sobre um determinado tema ou questão de pesquisa. O objetivo principal deste trabalho é explorar a Abordagem Lexical, enfocando suas implicações teóricas e práticas para o ensino de línguas. A análise se concentrará em como essa abordagem influencia os métodos de ensino, a aquisição de segunda língua e a preparação dos professores, além de investigar as críticas e as potencialidades identificadas na literatura especializada. Utilizaremos, em grande parte, o texto seminal do Michael Lewis, "The Lexical Approach", de 1993, bem como outros artigos e livros que nos ajudarão a explicitar o tema abordado. Ao elucidar estes aspectos, o trabalho visa contribuir para uma melhor compreensão de como a Abordagem Lexical pode ser aplicada de maneira eficaz no contexto educacional contemporâneo.


\section{CONCEITOS IMPORTANTES}

Antes de mergulharmos na exploração detalhada da Abordagem Lexical, é imperativo estabelecer uma fundação sólida sobre a qual possamos construir nossa discussão e análise. A compreensão das diversas teorias e práticas que informam o ensino de línguas não apenas contextualiza nossa investigação, mas também enriquece nosso entendimento das implicações práticas no campo do ensino de línguas. Portanto, nesta seção, delinearemos uma série de conceitos fundamentais que são pedras angulares no estudo da didática de línguas.

\subsection{ABORDAGEM, MÉTODO E TÉCNICA}

Em nossa argumentação vamos utilizar o modelo de classificação criada em 1963 pelo professor Edward Mason Anthony Jr. da Universidade de Michigan, que, segundo \citeonline{richards2001}, visava classificar uma desnorteadora variedade de termos usados para descrever as atividades nas quais se engajam e as crenças que possuem os professores de línguas. Em seu modelo Anthony propôs três importantes conceitualizações: a de abordagem, a de método e a de técnica.

Segundo Anthony uma abordagem é “um conjunto de suposições correlativas que lidam com a natureza do ensino e aprendizagem de línguas.” \cite[p. 19]{richards2001} A abordagem, ainda para Anthony, "é axiomática”. As abordagens estão, portanto, em nível mais estratégico. Segundo Richard e Rodgers (2001) as abordagens buscam seus fundamentos teóricos nas diferentes visões acerca da natureza da linguagem que os estudos linguísticos nos propõem. Em seu trabalho seminal, Lewis expande o conceito de abordagem e a apresenta como "é um conjunto integrado de crenças teóricas e práticas, que incorpora tanto o currículo quanto o método" \cite[p. 14]{lewis1993}.

Já o método, segundo Anthony \cite{richards2001}, “é um plano geral para a apresentação ordenada de material de linguagem, nenhuma parte do qual contradiz, e tudo baseado na abordagem selecionada.” O autor completa dizendo que “uma abordagem é axiomática, um método é procedimental” \cite[p. 19]{richards2001}. Podemos compreender que o papel do método nessa estrutura de elementos do ensino de línguas é menos estratégico e abstrato que o da abordagem, mas não tão tático e concreto quanto a técnica. Um método estrutura os princípios norteadores da abordagem e os converte em diretrizes práticas. Também nesse ponto \citeonline[p. 15]{lewis1993} apresenta o método como estratégias de sala de aula, dando a este um caráter mais prático que a abordagem.

E por último temos a técnica, que, como apresentado por \citeonline{richards2001}, “é implementacional - aquilo que realmente ocorre em uma sala de aula. É um truque, estratagema ou artifício específico usado para atingir um objetivo imediato. As técnicas devem ser consistentes com um método e, portanto, também em harmonia com uma abordagem.” Compreendemos, portanto, que a técnica é o processo tático per si, a atividade que é ministrada aos alunos e que carrega em si a estrutura do método, e este carrega em si os princípios da abordagem. Dessa forma, dizemos que os elementos dessa classificação apresentam-se de forma hierárquica, onde a abordagem fornece os pressupostos para o método, e este fornece a estrutura para a técnica.

Para finalizar esta seção, é crucial enfatizar que nosso foco subsequente será explorar a Abordagem Lexical primordialmente sob a perspectiva de uma abordagem, conforme definido por Edward Mason Anthony Jr. Dessa maneira, nosso objetivo não será detalhar técnicas individuais de aplicação ou métodos específicos, mas sim investigar e discutir os princípios teóricos e práticas integradas que fundamentam esta abordagem no ensino de línguas. Ao abordar a Abordagem Lexical como uma estrutura conceitual abrangente, pretendemos destacar como suas premissas orientam as práticas pedagógicas e influenciam a maneira como o ensino e a aprendizagem de línguas são percebidos e implementados. Esse entendimento mais estratégico nos permitirá avaliar a relevância e a eficácia da Abordagem Lexical no contexto contemporâneo do ensino de idiomas, contribuindo assim para uma compreensão mais profunda e aplicada deste influente modelo pedagógico.

\subsection{SOBRE A NATUREZA DA LÍNGUA}

Segundo \citeonline{richards2001}, pelo menos três visões teóricas da linguagem e sobre a natureza da proficiência em línguas implícita ou explicitamente embasam as diferentes abordagens e métodos de ensino de línguas. São elas: a visão estrutural, a visão funcional e a visão interacional.

Os autores as definem da seguinte forma:

\begin{citar}
A primeira, e a mais tradicional das três, é a visão estrutural, a visão de que a linguagem é um sistema de elementos estruturalmente relacionados para a codificação do significado. O alvo da aprendizagem de línguas é visto como o domínio dos elementos deste sistema (...). A segunda visão da linguagem é a visão funcional, a visão de que a linguagem é um veículo para a expressão do significado funcional. (...) Essa teoria enfatiza a dimensão semântica e comunicativa, em vez de meramente as características gramaticais da linguagem, e leva a uma especificação e organização do conteúdo do ensino de línguas por categorias de significado e função, e não por elementos de estrutura e gramática. (...) A terceira visão da linguagem pode ser chamada de visão interacional. Ela vê a linguagem como um veículo para a realização das relações interpessoais e para a realização das transações sociais entre os indivíduos. A linguagem é vista como uma ferramenta para a criação e manutenção das relações sociais. As áreas de investigação que estão sendo elaboradas no desenvolvimento de abordagens interacionais para o ensino de línguas incluem análise de interação, análise de conversação e etnometodologia. \cite{richards2001}
\end{citar}

No contexto deste trabalho, compreender as visões estrutural, funcional e interacional da linguagem é fundamental para situar a Abordagem Lexical dentro de um arcabouço teórico mais amplo sobre a natureza da língua e do ensino de idiomas. Essas três perspectivas nos ajudam a entender como diferentes abordagens e métodos de ensino refletem concepções variadas sobre o que é a linguagem e como ela deve ser aprendida e ensinada. Em particular, a Abordagem Lexical, ao enfatizar o aprendizado de "chunks" lexicais e a importância do léxico na comunicação, dialoga especialmente com a visão funcional, na qual a linguagem serve primariamente como um veículo para a expressão de significados funcionais e pragmáticos. Quando \citeonline[p. 101]{lewis1993} afirma que a fluência em uma língua é atingida em grande parte através da combinação de "chunks" para a diminuição da dificuldade de processamento por parte do falante podemos ver claramente essa conexão entre a visão funcional e a Abordagem Lexical. Mesmo assim, a abordagem em questão também incorpora elementos das visões estrutural e interacional, ao reconhecer a estruturação do léxico dentro da linguagem e sua utilização nas interações sociais, e isso é perceptível quando \citeonline[p. 32]{lewis1993} afirma que a Abordagem Lexical abraça tudo que a Abordagem Comunicativa sugere, com a adição da ênfase no léxico, fato também citado por \citeonline[p. 239]{torres_ramirez2012} que afirma que a Abordagem Lexical não é, de modo algum, uma quebra com a Abordagem Comunicativa, mas sim um desenvolvimento da mesma. Ao explorar a abordagem delineada por Lewis em nosso artigo, buscaremos destacar como ela se alinha e se diferencia dessas visões teóricas, proporcionando uma compreensão mais completa de sua aplicabilidade e eficácia no ensino de línguas.

\section{CONTEXTO HISTÓRICO}

Nos anos 90, o ensino de línguas enfrentava uma paisagem educacional em transformação, com mudanças significativas em relação às abordagens predominantes dos anos anteriores. Até então, o ensino de línguas estrangeiras era dominado pela abordagem gramatical e de tradução, onde as línguas modernas eram tratadas de maneira semelhante ao latim e ao grego clássico, com ênfase na compreensão de literatura na língua original e na maestria das estruturas gramaticais. No entanto, essa abordagem era criticada por sua separação artificial entre gramática e vocabulário, bem como pela falta de preparação dos alunos para o uso prático da língua. Essas críticas impulsionaram uma mudança em direção a abordagens mais centradas na comunicação, como a abordagem naturalista e a abordagem comunicativa, que ganharam destaque nas décadas de 1970 e 1980, respectivamente. Essas novas abordagens enfatizavam a aquisição e o uso significativo da língua, em contraste com a abordagem estrutural anterior, que priorizava a precisão gramatical. Essa transição refletiu uma crescente conscientização sobre a importância da comunicação eficaz em um mundo globalizado e impulsionou mudanças significativas no ensino de línguas nos anos 90.

No início dos anos 90, o ensino de línguas enfrentava desafios e oportunidades significativas devido às mudanças na concepção das metodologias de ensino. Com a crescente globalização e a necessidade de competências comunicativas eficazes em vários idiomas, o ensino de línguas viu um movimento em direção a abordagens mais integradas e práticas. Segundo \citeonline[p. ]{lewis1993} O ensino baseado em conteúdo, por exemplo, ganhou destaque como uma forma eficaz de aprofundar o domínio linguístico dos alunos enquanto eles aprendiam sobre outros assuntos, integrando o aprendizado da língua com o conteúdo acadêmico.  Crandall e Tucker, em \citeonline[p. 84]{anivan1990} afirmam que essa abordagem é "uma resposta às necessidades especiais de linguagem dos estudantes" \cite[p. 83]{anivan1990} e que, ao mesmo tempo, "busca desenvolver habilidades linguísticas acadêmicas necessárias para participar efetivamente de instrução de conteúdo​​" \cite[p. 84]{anivan1990}. Isso refletia uma mudança em relação aos métodos mais tradicionais focados apenas na gramática e vocabulário, movendo-se para uma abordagem que considerava a linguagem como um veículo para a expressão de ideias complexas e a realização de tarefas cognitivamente desafiadoras, onde programas baseados em conteúdo proporcionam aos alunos "a possibilidade de adquirir uma linguagem acadêmica mais formal, descontextualizada e cognitivamente complexa" \cite[p. 84]{anivan1990}.


Além disso, a necessidade de adaptar o ensino de línguas ao contexto cultural e às expectativas dos alunos também era cada vez mais reconhecida. Os educadores começaram a valorizar metodologias que levavam em consideração variáveis menos diretas e influências externas à sala de aula, como a cultura e os estilos de aprendizado dos alunos. Isso sugeria uma abordagem mais holística e sensível ao contexto para o ensino de línguas, que buscava não apenas ensinar o idioma de maneira eficaz, mas também engajar os alunos de uma maneira que respeitasse e valorizasse suas identidades culturais e necessidades individuais .

Portanto, o período inicial dos anos 90 foi marcado por uma busca contínua por metodologias de ensino que fossem tanto teoricamente sólidas quanto praticamente aplicáveis, refletindo um período de inovação e adaptação no campo do ensino de línguas, preparando o caminho para desenvolvimentos futuros em um mundo cada vez mais interconectado.

\section{PRINCÍPIOS CHAVE DA ABORDAGEM LEXICAL}

Ao desbravar o campo do ensino de línguas com sua obra seminal, Michael Lewis não apenas propôs uma nova perspectiva, mas também nos apresentou um conjunto de princípios fundamentais que servem como a espinha dorsal da Abordagem Lexical. Estes princípios não são apenas diretrizes; eles são revelações que desafiam e expandem nossa compreensão de como as línguas são aprendidas e ensinadas. Em sintonia com o método pioneiro de Lewis, esta seção de nosso trabalho buscará articular esses princípios chave, que, juntos, tecem o vasto e complexo tapeçar do léxico como o coração pulsante da aprendizagem da língua. \citeonline[p. 4]{torres_ramirez2012} define essa lista de princípios chave de "sumário prático de descobertas" das investigações de Lewis sobre a aquisição de primeira língua.

Ao adotarmos essa mesma estratégia elucidativa, nossa discussão não se limitará a uma mera exposição; aspiramos elevar a importância desses princípios à medida que exploramos suas implicações profundas para a prática pedagógica contemporânea. Portanto, tal como Lewis delineou os contornos de sua abordagem com princípios que são ao mesmo tempo profundos e práticos, nosso artigo se propõe a desdobrar essas noções-chave, estabelecendo um sólido entendimento da Abordagem Lexical e sua aplicação transformadora no ensino de línguas. Entre os vários princípios, pontuaremos os 4 que consideramos mais importantes para a compreensão da Abordagem Lexical como um todo.

\subsection{A linguagem consiste em léxico gramaticalizado, não em gramática lexicalizada}

O primeiro princípio proposto por Lewis é o de que a língua consiste em léxico gramaticalizado, e não em gramática lexicalizada \cite[p. 1]{lewis1993}. Esse princípio desafia a tradicional supremacia da gramática no ensino de idiomas, sugerindo que não são as regras gramaticais que dão forma ao léxico, mas sim o léxico que fundamenta e dá corpo à gramática. Em outras palavras, Lewis argumenta que o conhecimento de palavras e expressões idiomáticas — e como elas se combinam de formas que muitas vezes desafiam as regras gramaticais prescritivas — é o que realmente permite aos falantes se comunicarem efetivamente. O aprendizado da língua, portanto, como bem pontuado por \citeonline[p. 14]{smith2006} é inserir esses "chunks" na memória de longo prazo. O trabalho do professor que utiliza a Abordagem é facilitar o domínio de blocos construídos lexicalmente, ou "chunks", que já vêm com a gramática embutida neles, refletindo o uso autêntico e natural da língua, ao invés de uma sequência de palavras selecionadas para se encaixar em estruturas gramaticais predefinidas, como vemos na forma pedagógica das abordagens ligadas a gramática e a tradução. Isso implica um ensino que privilegia a exposição e a prática de frases prontas e estruturas comuns do discurso, permitindo que os alunos adquiram naturalmente as formas gramaticais juntamente com o léxico, de maneira mais orgânica e intuitiva.

\subsection{A dicotomia gramática/vocabulário é inválida; grande parte da linguagem consiste em "chunks" de várias palavras}

Este princípio postula que a língua não pode ser adequadamente dividida em gramática de um lado e vocabulário de outro, como se fossem entidades independentes e desconectadas. Em vez disso, Lewis defende que a língua é composta de 'chunks', ou seja, segmentos de múltiplas palavras que os falantes nativos frequentemente usam como unidades fixas de significado e função. Exemplos desses 'chunks' incluem expressões idiomáticas, colocações e estruturas de frase habituais, que não são meramente a soma de suas partes gramaticais e lexicais, mas sim entidades integradas que refletem o uso real da língua. \citeonline[p. 240]{torres_ramirez2012} exemplica esse princípio citando que "database management systems" é visto como uma unidade de informação ao invés de três. Dessa forma, Lewis redireciona o foco do ensino de línguas para a familiarização e prática desses blocos linguísticos, sugerindo que a competência comunicativa emerge mais eficazmente do domínio de tais construções lexicais do que da aprendizagem isolada de regras gramaticais ou listas de vocabulário. Este entendimento desafia a estrutura tradicional do ensino e aprendizagem de línguas, colocando a ênfase na aquisição de padrões linguísticos prontos e autênticos que são fundamentais para a fluência e a compreensão.

\subsection{Um elemento central do ensino de línguas é aumentar a consciencialização dos alunos e desenvolver a sua capacidade de “agrupar” a língua com sucesso}

Lewis coloca a consciência e habilidade dos alunos para "chunkar" a língua, ou seja, identificar e utilizar blocos lexicais, no cerne do processo de ensino de idiomas. Este princípio da Abordagem Lexical reconhece que a capacidade de decompor a linguagem em segmentos pré-fabricados e utilizá-los de maneira eficaz é essencial para a fluência linguística. Na prática, isso significa educar os alunos não apenas para reconhecer e memorizar palavras isoladas, mas para perceber as combinações habituais de palavras e as expressões que ocorrem naturalmente no discurso. Por exemplo, ensinar frases como "ask a question" ou "bear in mind" como unidades de significado completas, em vez de apenas as palavras isoladas "ask" ou "bear". \citeonline[p. 15]{smith2006} destaca a importância de que esse aumento da consciência ocorra de forma em que os estudantes deduzam tanto a forma como as regras gramaticais relativas à determinados "chunks". Essa consciência lexical implica uma compreensão de como os falantes nativos agrupam palavras em padrões reconhecíveis, que são recuperados como um todo durante a comunicação. No que tange à transformação da Abordagem Lexical em método, Lewis aponta que "se os alunos forem expostos a dados de linguagem natural (habilidades receptivas), apresentados a tarefas e perguntas apropriadas (aumento da conscientização) e sua percepção das diferenças for aumentada (monitoramento eficaz), as 'explicações' que os alunos gerarão serão intensamente pessoais e totalmente internalizadas." \cite[p. 151]{lewis1993}. Dessa forma conseguimos compreender o lugar que o aumento da consciência toma dentro da Abordagem Lexical, como um princípio norteador importante e inalienável no processo de aquisição de segunda linguagem.


\subsection{A competência sociolinguística – poder comunicativo – precede e é a base, e não o produto, da competência gramatical}

Para Lewis a capacidade de comunicar eficazmente em situações sociais reais — a competência sociolinguística — é o fundamento inicial e essencial no processo de aprendizagem de uma língua, e não um resultado da competência gramatical. Lewis argumenta que a competência comunicativa, que ele comumente chama de "poder comunicativo" possui centralidade na Abordagem Comunicativa: "O tempo filosófico, psicológico e histórico tem razão em enfatizar o empoderamento do indivíduo e, na linguagem, a centralidade do poder comunicativo." \cite[p. 74]{lewis1993}. Assim, ao contrário de abordagens mais tradicionais que colocam a gramática no centro do ensino de línguas, a Abordagem Lexical de Lewis sugere que a fluência e a capacidade de comunicação devem ser priorizadas, servindo como alicerce para o aprimoramento subsequente da precisão gramatical.

Através da exposição e explicação destes princípios buscamos elucidar não somente a estrutura teórica por trás do método, mas também a aplicação prática que redefine o processo de ensino e aprendizado de línguas. Eles apresentam o cerne da Abordagem Lexical como uma prática pedagógica que visa reorientar o foco do ensino para o léxico como base da competência comunicativa, que, segundo \citeonline[p. 19]{smith2006} promove a apreensão mais objetiva e natural de uma linguagem, além de acelerar o processo de aprendizado.

\section{A NATUREZA DO LÉXICO E OS PRINCIPAIS ITENS LEXICAIS}


Na Abordagem Lexical, o léxico não é apenas um componente do ensino de línguas; ele é o cerne do aprendizado e da aquisição linguística. Esta perspectiva, sustentada por \citeonline[p. 9]{lewis1993}, que ressalta que a gramática deve ser reconhecida como um aprendizado passivo que emerge como um subproduto do aprendizado lexical, e não como seu precursor. \citeonline[p. 242]{torres_ramirez2012} aponta acertamente que um crescente corpo literário que estuda aquisição de segunda língua tem conectado a fluência em L2 ao domínio de "um grande número de sequências padronizadas ou ritualizadas", e estabelecendo como corolário que há ligação entre fluência falada e competência lexical. A Abordagem Lexical desloca o foco tradicional da gramática para uma concentração no léxico, em consonância com vasta essa vasta literatura que, desde a introdução desta abordagem na década de 90, tem dado suporte aos seus princípios.

Nesse contexto da centralidade do léxico no aprendizado de L2 surge um conceito chave para a Abordagem Lexical: os itens lexicais. Os itens lexicais são as unidades fundamentais do léxico, essenciais para a compreensão e prática da Abordagem Lexical. \citeonline[p. 90]{lewis1993} afirma que os itens lexicais são unidades independentes "sancionadas socialmente". "Em vez de confiar no poder gerador da gramática, os usuários contam com um vasto estoque de frases fixas e locuções pré-padronizadas pelas quais gerenciam rotineiramente aspectos da interação." \cite[p. 90]{lewis1993}.

Apresentaremos a seguir os itens lexicais mais importantes para a compreensão da Abordagem Lexical.

\subsection{Palavras}

\subsection{Multi-word items}
\subsection{Polywords}
\subsection{Collocations}
\subsection{Expressões institucionalizadas}

\section{A ABORDAGEM COMO MÉTODO}

\section{CONCLUSÃO}


  \section{REFERÊNCIAS}

  \bibliography{bibliografia.bib}
  \bibliographystyle{abntex2-alf} % Estilo de bibliografia ABNT alfabético

\end{document}
