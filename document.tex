\documentclass[        
    a4paper,          % Tamanho da folha A4
    12pt,             % Tamanho da fonte 12pt
    section=TITLE,    % Todas as secoes devem ter caixa alta
    subsection=Title, % Todas as subsecoes devem ter caixa alta somente na primeira letra
    oneside,          % Usada para impressao em apenas uma face do papel
    english,          % Hifenizacoes em ingles
    spanish,          % Hifenizacoes em espanhol
    brazil,           % Ultimo idioma eh o idioma padrao do documento
    fleqn             % Comente esta linha se quiser centralizar as equacoes. Comente também a linha 65 abaixo
]{abntex2}

\counterwithout{section}{chapter}

% Importações de pacotes
\usepackage[utf8]{inputenc}                         % Acentuação direta
\usepackage[T1]{fontenc}                            % Codificação da fonte em 8 bits
\usepackage{graphicx}                               % Inserir figuras
\usepackage{amsfonts, amssymb, amsmath}             % Fonte e símbolos matemáticos
\usepackage{booktabs}                               % Comandos para tabelas
\usepackage{verbatim}                               % Texto é interpretado como escrito no documento
\usepackage{multirow, array}                        % Múltiplas linhas e colunas em tabelas
\usepackage{indentfirst}                            % Endenta o primeiro parágrafo de cada seção.
\usepackage{listings}                               % Utilizar codigo fonte no documento
\usepackage{xcolor}
\usepackage{microtype}                              % Para melhorias de justificação?
\usepackage[portuguese,ruled,lined]{algorithm2e}    % Escrever algoritmos
\usepackage{algorithmic}                            % Criar Algoritmos  
%\usepackage{float}                                 % Utilizado para criação de floats
\usepackage{amsgen}
\usepackage{lipsum}                                 % Usar a simulação de texto Lorem Ipsum
\usepackage{titlesec}                              % Permite alterar os títulos do documento
\titleformat*{\section}{\normalfont\Large\bfseries}
\usepackage{tocloft}                                % Permite alterar a formatação do Sumário
\usepackage{etoolbox}                               % Usado para alterar a fonte da Section no Sumário
\usepackage[nogroupskip,nonumberlist]{glossaries}   % Permite fazer o glossario. A apcao "sort=use" faz com que as siglas aparecam na lista conformse sao usadas no texto.

\usepackage[format=plain,justification=justified,skip=0pt,singlelinecheck = false,labelsep=colon]{caption}            % Altera o comportamento da tag caption. Algumas opcoes do caption so podem ser alternada no arquivo "antex2.cls, linhas 334 a 348.

%\usepackage[bottom]{footmisc}                      % Mantém as notas de rodapé sempre na mesma posição
%\usepackage{times}                                 % Usa a fonte Times
%%%%%%%%%%%%%%%%%%% AVISO %%%%%%%%%%%%%%%%%%%%%%%%%%%%%%%%%%%%%%%%
%descomente as duas linhas abaixo para alterar o texto de Times New Roman para Arial:

%\usepackage{helvet}
%\renewcommand{\familydefault}{\sfdefault}  % Usa a fonte Arial              
%%%%%%%%%%%%%%%%%%%%%%%%%%%%%%%%%%%%%%%%%%%%%%%%%%%%%%%%%%%%%%%%%%

\usepackage{mathptmx}         % Usa a fonte Times New Roman			
%\usepackage{lmodern}         % Usa a fonte Latin Modern
%\usepackage{subfig}          % Posicionamento de figuras
%\usepackage{scalefnt}        % Permite redimensionar tamanho da fonte
%\usepackage{color, colortbl} % Comandos de cores
%\usepackage{lscape}          % Permite páginas em modo "paisagem"
%\usepackage{ae, aecompl}     % Fontes de alta qualidade
%\usepackage{picinpar}        % Dispor imagens em parágrafos
%\usepackage{latexsym}        % Símbolos matemáticos
%\usepackage{upgreek}         % Fonte letras gregas
\usepackage{appendix}         % Gerar o apendice no final do documento
\usepackage{paracol}          % Criar paragrafos sem identacao
\usepackage{pdfpages}         % Incluir pdf no documento
\usepackage{amsmath}          % Usar equacoes matematicas

\makeglossaries % Organiza e gera a lista de abreviaturas, simbolos e glossario
\makeindex      % Gera o Indice do documento         

\renewcommand{\labelitemi}{\textendash} %Altera os marcadores de itemize para 

\title{O Léxico Gramatizado: Uma Exploração Bibliográfica da Abordagem Lexical}
\author{André Carvalho} 
\date{2024\\ Abril}


\begin{document}	

  \begin{flushright}
    \textbf{André Carvalho}

    \textbf{Nome do Professor}
  \end{flushright}

  \begin{center}
      \Large \textbf{RESUMO}
  \end{center}

  \noindent // TODO Aqui é apresentado o trabalho de forma sucinta (até 250 palavras, de 8 a 10 linhas, apenas um parágrafo e espaçamento simples), informando ao leitor finalidade, metodologia, resultados e conclusão do documento, de tal forma que este possa, inclusive, dispensar a consulta ao original se julgado que o conteúdo não é de seu interesse. O resumo deve ser composto de uma sequência de frases concisas, afirmativas, e não de enumeração de tópicos, lembrando que é necessário o uso de parágrafo único. A primeira frase deve ser significativa, explicando o tema principal do documento. A seguir, deve-se indicar a informação sobre a categoria do tratamento (memória, estudo de caso, análise da situação etc.).

  \noindent \textbf{PALAVRAS-CHAVE:} Termos e Palavras 1. Termos e Palavras 2. Termos e Palavras 3.

  \section{INTRODUÇÃO}

  A história do ensino de línguas, como aponta Richards e Rodgers (2001, p. 3), é caracterizada pela busca de formas mais eficientes de ensinar uma segunda língua. Embora possa-se dizer que esse campo de conhecimento possui séculos de existência — inicialmente dominados pela tradição que herdou do ensino de latim os métodos e procedimentos de ensino de línguas — foi, segundo Richard e Rodgers (2001, p. 15, tradução nossa) no período entre os anos 1950 e 1980 que a busca por formas mais eficientes de ensinar uma segunda língua intensificou-se. Isso se deu, em grande parte, pela importância que a língua inglesa adquiriu como meio de instrução na sociedade nesse período, marcada por um intenso processo de globalização e cujas comunicações foram amplamente transformadas pelo advento da Tecnologia da Informação e Comunicação (TIC), aumentando portanto a demanda por eficiência nesse ramo. As transformações nos campos da linguística, linguística aplicada, psicologia do desenvolvimento e psicologia da educação também foram responsáveis pelo embasamento teórico de uma série de novas formas de pensar o ensino de línguas, além de retirar tração e suportar antigas formas de pensar esse mesmo campo. A união desses aspectos supracitados fomentou então um contexto ideal para o surgimento de novas formas de realizar a prática docente na área de línguas. 

  Nesse contexto do surgimento de novos métodos de ensino de línguas surgiu a Abordagem Lexical, desenvolvida por Michael Lewis no início dos anos 1990. O texto seminal de Lewis surgiu como uma revolução no campo da linguística aplicada, sobretudo no ensino de língua estrangeira.

\end{document}
